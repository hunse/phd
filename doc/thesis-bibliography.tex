% B I B L I O G R A P H Y
% -----------------------

\cleardoublepage
\phantomsection
\renewcommand*{\bibname}{References}
\addcontentsline{toc}{chapter}{\textbf{References}}
\printbibliography


%% % The following statement selects the style to use for references.  It controls the sort order of the entries in the bibliography and also the formatting for the in-text labels.
%% %% \bibliographystyle{plain}
%% \bibliographystyle{ieeetr}

%% % This specifies the location of the file containing the bibliographic information.
%% % It assumes you're using BibTeX (if not, why not?).
%% \cleardoublepage % This is needed if the book class is used, to place the anchor in the correct page,
%%                  % because the bibliography will start on its own page.
%%                  % Use \clearpage instead if the document class uses the "oneside" argument
%% \phantomsection  % With hyperref package, enables hyperlinking from the table of contents to bibliography
%% % The following statement causes the title "References" to be used for the bibliography section:
%% \renewcommand*{\bibname}{References}

%% % Add the References to the Table of Contents
%% \addcontentsline{toc}{chapter}{\textbf{References}}

%% \bibliography{/media/dropbox/papers/bib/library}
%% % Tip 5: You can create multiple .bib files to organize your references.
%% % Just list them all in the \bibliogaphy command, separated by commas (no spaces).

%% % The following statement causes the specified references to be added to the bibliography% even if they were not
%% % cited in the text. The asterisk is a wildcard that causes all entries in the bibliographic database to be included (optional).
%% %% \nocite{*}
